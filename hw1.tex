\documentclass[12pt]{article}
\usepackage{listings}
\usepackage{graphicx} % for including gif

% The following nonsense is coloring for code samples %%%%%%%
\usepackage{color}

\definecolor{codegreen}{rgb}{0,0.6,0}
\definecolor{codegray}{rgb}{0.5,0.5,0.5}
\definecolor{codepurple}{rgb}{0.58,0,0.82}
\definecolor{backcolour}{rgb}{0.95,0.95,0.92}

\lstdefinestyle{mystyle}{
	backgroundcolor=\color{backcolour},   
	commentstyle=\color{codegreen},
	keywordstyle=\color{magenta},
	numberstyle=\tiny\color{codegray},
	stringstyle=\color{codepurple},
%	basicstyle=\footnotesize,
	breakatwhitespace=false,         
	breaklines=true,                 
	captionpos=b,                    
	keepspaces=true,                 
	numbers=left,                    
	numbersep=5pt,                  
	showspaces=false,                
	showstringspaces=false,
	showtabs=false,                  
	tabsize=2
}

\lstset{style=mystyle}
%%%%%%%%%%%%%%%%%%%%%%%%%end nonsense%%%%%%%%%%%%%%%%%%%%%%%%%%%%%%%%%%%%%%

\begin{document}
\section*{Problem 1}
\begin{itemize}
	\item $P_t$ is the amount of principal in the account at time $t$. $t$ ranges over the reals.
	\item $P_0$ is the initial principal in the account. $P_0$ ranges over the reals.
	\item $m$ is the number of times the interest is compounded per year. $m$ must be positive.
	\item $j$ is the annual interest rate. One would think that interest must be positive. The recent financial history of the European Union shows that $j$ ranges over the reals.
	\item $t$ is the time in years that the account has been growing. $t$ ranges over the reals.
\end{itemize}

\section*{Problem 2}
I asked in class if you meant that the interest is compounded monthly or yearly here, and you answered yearly. So I proceed to answer accordingly.

$$P_5 = \$1000(1+\frac{0.015}{1})^{5\times 1} = \$1077.28$$

\section*{Problem 3}
\begin{itemize}
	\item $P_5 = \$1000(1+\frac{0.015}{2})^{5\times 2} = \$1077.58$
	\item $P_5 = \$1000(1+\frac{0.015}{4})^{5\times 4} = \$1077.73$
	\item $P_5 = \$1000(1+\frac{0.015}{6})^{5\times 6} = \$1077.78$
	\item $P_5 = \$1000(1+\frac{0.015}{12})^{5\times 12} = \$1077.83$
\end{itemize}

As the frequency with which the interest compounds increases, the total value of the account goes up over the same time period.

\section*{Problem 4}


\begin{lstlisting}[language=python, caption=$P_t$ at different values of $m$]
1 #!/usr/bin/python3.5
2 P_0 = 1000
3 t = 5
4 j = 0.015
5 
6 # iterate over values of m, and print corresponding P_t
7 for m in range(1,100):
8     P_t = P_0*(1+j/m)**(t*m)
9     print("{} {}".format(m,P_t))
\end{lstlisting}

\includegraphics[]{graph4.png}


\end{document}